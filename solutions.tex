%CS-113 S18 HW-3
%Released: 16-Feb-2018
%Deadline: 2-March-2018 7.00 pm
%Authors: Abdullah Zafar, Waqar Saleem.


\documentclass[addpoints]{exam}

% Header and footer.
\pagestyle{headandfoot}
\runningheadrule
\runningfootrule
\runningheader{CS 113 Discrete Mathematics}{Homework III}{Spring 2018}
\runningfooter{}{Page \thepage\ of \numpages}{}
\firstpageheader{}{}{}

\boxedpoints
\printanswers
\usepackage[table]{xcolor}
\usepackage{amsfonts,graphicx,amsmath,hyperref,amsthm}

\title{Habib University\\CS-113 Discrete Mathematics\\Spring 2018\\HW 3 Solutions}
\author{}  % replace with your ID, e.g. oy02945
\date{Released: 6th March, 2018}


\begin{document}
\maketitle

\begin{questions}



\question
All sets carry data, but how much information can be extracted from it? Consider a simple model on a set $A$, in which each relation encodes 1 unit of information. We define the ``Information Potential" of a set as the sum of information units (or the number of distinct relations) that can be generated from the set. In the questions that follow, you may assume all relations to be binary.

\begin{parts}
  \part Consider $A$ to be the set of $n$ distinct facts. What is the information potential of this set?
  
  \begin{solution}
    $2^{n^2} | $ There are $n^2$ bits to flip in the matrix representation of $A \times A$.
  \end{solution}
  
  \part Reflexive pairs of the form $(fact\;x, fact\;x)$ are considered redundant in our model. What is the information potential of the ``non-redundant" set, that is, the set without reflexive relations? 
  
  \begin{solution}
    $2^{n^2 - n}(2^n - 1) |$ The $n$ bits of the leading diagonal may be flipped to all configurations except all 1's ($2^n -1$ combinations). The remaining $n^2 - n$ bits can be flipped freely.
  \end{solution}

  \part Anti-symmetric relations that follow the rule $(fact\;x,fact\;y)\; \land (fact\;y,fact\;x) \rightarrow fact\;x = fact\;y$ are of special interest to our model. Such pairs, as in the aforementioned antecedent, can be used to express ordered relationships between facts. What is the combined Information Potential of anti-symmetric relations on the non-redundant set? 
  \begin{solution}
    $2^{(n^2-n)/2}(2^n - 1)| $The $n$ bits of the leading diagonal may be flipped to all configurations except all 1's ($2^n -1$ combinations). Only half of the remaining $n^2 - n$ bits can be flipped freely to preserve anti-symmetry.
    
  \end{solution}
  
  \part There are many ways to describe a relation in natural language. For example, a relation described as $``x<y"$ over the set $\{1,2\}$ may also be described as $``x+1=y"$. Specifically, two descriptions that produce the same relation are considered ``isomorphs" of one another in our model. There may be any number of isomorphs for a given relation. Given $2^{n^2+1}$ descriptions of relations, how many isomorphs exist? (Give your answer as a range)
  
  \begin{solution}
    $(2^{n^2} + 1, 2^{n^2+1})|$ There are $2^{n^2}$ possible relations on A, therefore by the pigeonhole principle, at least $2^{n^2}$ non-unique descriptions exist. If these descriptions are all isomorphic to each other, then there are a total of $2^n + 1$ isomorphs, giving us the lower bound. The upper bound is simply the total number of descriptions, in the case when all are isomorphic to each other. \textit{Note}: Isomorphs exist as collections of size 2 or more, i.e. there is no such thing as a standalone isomorph.
  \end{solution}

\end{parts}

\question Let $R$ be a relation from $A$ to $B$. Then the inverse of $R$, written $R^{-1}$, is a relation from $B$ to $A$ defined by $R^{-1} = \{(y,x) \in B \times A \:|\: (x,y) \in R\}$. Prove that $R$ is symmetric iff $R = R^{-1}$.

  \begin{solution}
  \\
   To prove that \textit{$R$ is symmetric $\rightarrow R = R^{-1}$}:\\
   Since $R$ is symmetric, $\forall a,b((a,b) \in R \rightarrow (b,a) \in R)$. Consider some $(a,b) \in R$. By definition of symmetry, $(b,a) \in R$. But by definition of inverse, $(b,a) \in R^{-1}$ as well. Therefore, $R \subseteq  R^{-1}$. By a symmetric (pun not intended) argument, $R^{-1} \subseteq R$. Therefore, $R = R^{-1}$.\\\\
   To prove that \textit{$R = R^{-1} \rightarrow R$ is symmetric}:\\
   Suppose $R = R^{-1}$. Consider some $(a,b) \in R$. By definition of inverse, $(b,a) \in R^{-1}$. But since $R = R^{-1}, (b,a) \in R$. Thus, $R$ is symmetric.\qed
  \end{solution}

\question Let $R$ and $S$ be relations on a set $A$. Assuming $A$ has at least 3 elements, state whether each of the following statements is true or false, providing a brief explanation if true, or a counterexample if false:
\begin{parts}
\part If $R$ and $S$ are reflexive, then $R \cup S$ is reflexive.
\part If $R$ and $S$ are anti-symmetric, then $R \circ S$ is anti-symmetric.
\part If $R$ and $S$ are symmetric, then $R \cap S$ is symmetric.
\part If $R$ is reflexive, then $R \cap R^{-1}$ is not empty. 
\part If $R$ is transitive, then $R^{-1}$ is transitive.
\end{parts}


  \begin{solution}
    \begin{parts}
    \part True $|$ The union contains all reflexive pairs because it contains all members of $R$ (same could be said with $S$).  
    \part False $|$ Let $R = \{(1,2),(3,4)\}$ and $S=\{(2,3),(4,1)\}$, then $R \circ S = \{(1,3),(3,1)\}$ which is not anti-symmetric. 
    \part True $|$ The intersection will contain elements present in $R$ and $S$. But $R$ and $S$ contain symmetric pairs only.
    \part True $|$ The intersection contains pairs of the form $(a,a) \in A$, hence is non-empty.
    \part True $|$ Let $(a,b) \in R$ and $(b,c) \in R$. Since $R$ is transitive, $(a,c) \in R$. By definition of inverse, each of $(c,b),(b,a), (c,a) \in R^{-1}$, hence it is transitive too.
    \end{parts}
  \end{solution}
  
\question Let $R$ be a relation on $A$. Prove that the digraph representation of $R$ has a path of length $n$ from $a$ to $b$ iff $(a, b) \in R^n$.

  \begin{solution}
    \textit{Note}: We proceed via induction on $n$.  Since this is an `iff' statement with $n$ on both sides, it is convenient to proceed with a bi-conditional proof here.
    
    To prove that \textit{$R$ has a path of length $n$ from $a$ to $b$ iff $(a, b) \in R^n$}:
    
    \textbf{Base Case, n=1}: $R$ has a path of length $1$ from $a$ to $b$ iff $(a, b) \in R$.
    
    \textbf{Inductive Step}: To prove \textit{$R$ has a path of length $n$ from $a$ to $b$ iff $(a, b) \in R^n \rightarrow R$ has a path of length $n+1$ from $x$ to $z$ iff $(x, z) \in R^{n+1}$}.
    
    Consider a path of length $n+1$ from $x$ to $z$ in the digraph of $R$. Since $n+1 \geq 2$, there exists some node $y$, such that $x$ reaches $y$ and $y$ reaches $z$. Without loss of generality, let the path length of $(x,y) = 1$ and that of $(y,z) = n$. By Inductive assumption, $((y,z) \circ (x,y) \in R^n \circ R) \equiv (x,z) \in R^{n+1}$. This concludes the `only if' direction of the proof. \\
    Conversely, for the `if' part, let $(x,z) \in R^{n+1}$. Then, by definition of composition, $\exists y((x,y) \in R \wedge (y,z) \in R^n)$. Applying the inductive assumption here gives us a path of length $n$ from $y$ to $z$. Therefore, the length of $(x,z) =$ length of $(x,y)$ + length of $(y,z) = 1 + n$. \qed   
    \end{solution}

\question
    Let $R$ be a relation on a set $A$. We define

    $\rho (R) = R \cup \{(a, a) | a \in A\}$ \\ 
    $\phi (R) = R \cup R^{-1}$ \\
    $\tau (R) = \cup \{ R^n | n = 1,2,3,...\}$
    
    Show that $\tau (\phi (\rho (R)))$ is an equivalence relation containing $R$.
    
      \begin{solution}
    By definition, $\tau (\phi (\rho (R)))$ 
    \begin{align}
    &= \cup\{((R \cup \{(a,a) | a \in A\}) \cup (R \cup \{(a,a) | a \in A\})^{-1})^n | n = 1,2,3,...\}\\
    &= \cup\{(R \cup R^{-1} \cup \{(a,a) | a \in A\})^n | n = 1,2,3,...\}\\
    &= \cup\{S^n | n=1,2,3,...\} \text{ and } S=(R \cup R^{-1} \cup \{(a,a) | a \in A\}) 
    \end{align}
    To prove equivalence, we show that (3) is:
        \begin{subparts}
        \subpart Reflexive: The union of a reflexive set with other sets preserves reflexiveness and S contains the reflexive set $\{(a,a) | a \in A\}$. Therefore (3) contains the reflexive set as well.
         
        \subpart Symmetric: By definition of inverse, $\forall a,b ((b,a) \in R^{-1}\;|\;(a,b) \in R)$, and S contains both $R$ and $R^{-1}$. Therefore, $S$ is symmetric. Furthermore, if $S$ is symmetric, then $S^p, p \geq 1$ is symmetric$^1$, and the union of symmetric sets is symmetric. Therefore, (3) is symmetric. 
        
        \subpart Transitive: Consider $(a,b) \in (3)$ and $(b,c) \in (3)$. Without loss of generality, let $(a,b) \in S^p$ and $(b,c) \in S^q, p,q \geq 1$. Then, $(a,c) \in S^{p+q}$, which is contained in (3) as well. Therefore (3) is transitive.
        \qed 
        
        \end{subparts}
        $^1$ Here's the interesting part of the inductive proof that \textit{$S$ is symmetric $\rightarrow S^p$ is symmetric}: Let $S^p$ be symmetric. Consider $(a,c) \in S^{p+1}$. Then there exists $b$, such that $(a,b) \in S$ and $(b,c) \in S^p$. Since $S$ and $S^p$ are both symmetric (Inductive assumption), $(b,a) \in S$, and $(c,b) \in S^p$. Therefore $(c,a) \in S \circ S^p = S^p \circ S = S^{p+1}$. 
  \end{solution}


\end{questions}

\end{document}